\documentclass[onecolumn,apj]{emulateapj}
\usepackage{ctable}
\usepackage{amsmath}
\usepackage{graphicx}
\usepackage{hyperref}
\usepackage{mathrsfs}
%\usepackage[figuresright]{rotating}
%\usepackage{rotating}
\usepackage{natbib}
%\usepackage{pdflscape}
%\usepackage{lscape}
%\citestyle{aa}

\definecolor{applegreen}{rgb}{0.55, 0.71, 0.0}
\newcommand{\mep}[1]{{\color{applegreen} \textbf{[MEP:  #1]}}}

\begin{document}

\title{Senior Thesis Draft}
\author{Morgan Presley}

\section{General Outline of Thesis}
\begin{itemize}
	\item Introduction \& Background Theory
		\begin{itemize}
			\item Story of Inflation
			\item Motivation (Why do we need inflation)
			\item Original failure of Guth's inflation
			\item Slow roll inflation
			\item Problems with current theory (domination of young bubbles; infinite multiverse)
			\item Current status: trying to impose a measure to make bubbles like us more common
			\item But let's ignore the current problems and I'll show that even the simplest models in the current theory must be very fine-tuned to reproduce the current data
		\end{itemize}
	\item How to quantize complexity / fine tuning
		\begin{itemize}
			\item Look at Latham's paper
		\end{itemize}
	\item Examine tuning of inflationary models
		\begin{itemize}
			\item My reproduction of Latham's results
		\end{itemize}
	\item Examine tuning of ekpyrotic models
		\begin{itemize}
			\item My results
		\end{itemize}
	\item Examine tuning of anamorphic models (maybe)
		\begin{itemize}
			\item Have to wait until paper comes out
		\end{itemize}	
	\item Conclusion
\end{itemize}

\end{document}