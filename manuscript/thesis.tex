\documentclass[onecolumn,apj]{emulateapj}
\usepackage{ctable}
\usepackage{amsmath}
\usepackage{graphicx}
\usepackage{hyperref}
\usepackage{mathrsfs}
%\usepackage[figuresright]{rotating}
%\usepackage{rotating}
\usepackage{natbib}
%\usepackage{pdflscape}
%\usepackage{lscape}
%\citestyle{aa}

\def\d{\mathrm{d}}

\definecolor{applegreen}{rgb}{0.55, 0.71, 0.0}
\newcommand{\mep}[1]{{\color{applegreen} \textbf{[MEP:  #1]}}}

\begin{document}

\title{Senior Thesis Draft}
\author{Morgan Presley}

\section{General Outline of Thesis}
\begin{itemize}
	\item Introduction \& Background Theory
		\begin{itemize}
			\item Story of Inflation
			\item Motivation (Why we need inflation)
			\item Original failure of Guth's inflation
			\item Slow roll inflation
			\item Problems with current theory (domination of young bubbles; infinite multiverse)
			\item Current status: trying to impose a measure to make bubbles like us more common
			\item But let's ignore the current problems and I'll show that even the simplest models in the current theory must be very fine-tuned to reproduce the current data
		\end{itemize}
	\item How to quantize complexity / fine tuning
		\begin{itemize}
			\item Look at Latham's paper
		\end{itemize}
	\item Examine tuning of inflationary models
		\begin{itemize}
			\item My reproduction of Latham's results
		\end{itemize}
	\item Examine tuning of ekpyrotic models
		\begin{itemize}
			\item My results
		\end{itemize}
	\item Examine tuning of anamorphic models (maybe)
		\begin{itemize}
			\item Have to wait until paper comes out
		\end{itemize}	
	\item Conclusion
\end{itemize}

\section{Introduction}
Inflationary theory was first proposed by Alan Guth as a possible solution to the horizon and flatness problems \citep{Guth1981}. At the time, the standard cosmological model of the early universe was an adiabatically expanding, radiation-dominated universe. This model had two fundamental problems: the horizon problem and the flatness problem. Simply put, the horizon problem is the realization that the universe is homogenous and isotropic on scales far larger than the particle horizon (the farthest distance from which a point could ever receive information, given the age and expansion rate of the universe). 

\section{Mathematical Derivations (Not a real section, just a holding place for stuff)}
\subsection{Inflationary Friedmann equation}
To see how one could implement an inflationary theory, let us first examine a non-inflationary Friedmann equation:
\begin{equation}
H^2 = \left ( \frac{\dot a(t)}{a(t)} \right ) ^2 = \frac{8 \pi G}{3} \left [ \frac{\rho^0_m}{a^3} + \frac{\rho^0_r}{a^4} + \Lambda + \frac{\sigma^2}{a^6} + \frac{K}{a^2} \right ]
\end{equation}
where $\rho^0_m$ is the energy density of matter in the universe at time $t=0$ (set to be the present time), $\rho^0_r$ is the current energy of radiation, $\Lambda$ is the cosmological constant, $\sigma$ is the cosmic anisotropy term, and $K$ is the curvature of the universe. As we can see by examining the powers of $a$ in each term, the early universe is dominated by the anisotropy term, $\propto a^{-6}$, while the late universe is dominated by the curvature term (and cosmological constant). Therefore, the simplest way to add inflation to the model of the universe is to add an inflationary term to the Friedmann equation: 
\begin{equation}
H^2 = \left ( \frac{\dot a(t)}{a(t)} \right ) ^2 = \frac{8 \pi G}{3} \left [ \frac{\rho^0_m}{a^3} + \frac{\rho^0_r}{a^4} + \Lambda + \frac{\sigma^2}{a^6} + \frac{K}{a^2} + \frac{\rho^0_\phi}{a^{2\epsilon}} \right ]
\end{equation}
where $\phi^0_\phi$ is the energy density of the inflationary field and $\epsilon$ is a positive exponent that must satisfy $\epsilon>3$ for inflation to dominate in the early universe. \mep{I have in my notes that $\epsilon<1$ for inflation and $\epsilon>3$ for contraction, but that doesn't make sense, so I think that's wrong.} 

Now using this form for the inflationary term, let's examine what an inflation-dominated universe would look like. In a $\phi$-dominated universe, the Friedmann equation becomes:
\begin{equation}
\left ( \frac{\dot a}{a} \right ) ^2 = \frac{H_0 \Omega_\phi}{a^{2\epsilon}},
\end{equation}
where $H_0$ is the Hubble constant and $\Omega_\phi = \tfrac{8\pi G \rho_\phi}{3 H_0}$. Now, solving this equation for $a(t)$ gives
\begin{equation}
a(t) \propto (\epsilon \sqrt{H_0 \Omega_\phi} t)^{1/\epsilon},
\end{equation}
and hence 
\begin{equation}
\ddot a(t) \propto \left ( \frac{1-\epsilon}{\epsilon} \right ) (H_0 \Omega_\phi)^{1/2\epsilon} t^{\tfrac{1-2\epsilon}{\epsilon}}. 
\end{equation}
From this, we can see that the condition for an expanding universe is $\epsilon<1$. \mep{Shoot. So my notes must have been right. But how can a small power in the denominator dominate at small values of a? That doesn't make sense!?!?} Therefore, we want to find an inflationary theory that produces an inflationary term in the Friedmann equation $\propto a^{-2\epsilon}$ so that the universe will be dominated by expansion at early times. In order to derive a field theory that will produce such an inflationary term, we turn to the equation of state, $w=\rho/p$, where $\rho$ is the energy density and $p$ is the pressure of the field. The equation of state is related to $\epsilon$ by the equation $\epsilon = \tfrac{3}{2}(w+1)$. Table \ref{tab:field_scenarios} summarizes the values $w$ and $\epsilon$ and the evolution of $a(t)$ for different types of fields. 

\begin{table}[htbp]
   \centering
   \begin{tabular}{@{} lcccc @{}} % Column formatting, @{} suppresses leading/trailing space
      \toprule
      Field Type & $w=\rho/p$ &  $\epsilon=\tfrac{3}{2}(w+1)$ & $a(t)$ & $\ddot a(t)$ \\
      \midrule
      matter & 0 & $3/2$ & $\propto t^{2/3}$ & $>0$ \\
       radiation & $1/3$ & $2$ & $\propto t^{1/2}$ & $<0$ \\
       $\Lambda$ & $-1$ & $0$ & $\propto e^t$ & $>0$ \\
       inflation & $<-1/3$ & $<1$ & $\propto t^{1/\epsilon}$ & $>0$ \\
       contraction & $>1$ & $>3$ & $\propto t^{1/\epsilon}$ & $<0$ \\
      \bottomrule
   \end{tabular}
   \caption{The values $w$ and $\epsilon$ and the evolution of $a(t)$ for different types of fields.}
   \label{tab:field_scenarios}
\end{table}


\subsection{Inflationary Field}
Let $\phi$ be an inflationary field. It is governed by two equations, the Friedmann equation: 
\begin{equation}
H = \sqrt{\tfrac{1}{3} \left ( \tfrac{1}{2} \dot \phi^2 + V(\phi) \right )},
\label{eqn:Friedmann}
\end{equation}
and the Klein-Gordon equation:
\begin{equation}
\ddot \phi + 3H \dot \phi + V'(\phi) = 0,
\label{eqn:KG}
\end{equation}
where $H$ is the Hubble parameter, $V(\phi)$ is the inflationary potential, and a dot indicates a time derivative. In our convention, the reduced Planck mass is set to one, i.e. $M_{pl} = (8\pi G)^{-1/2} \equiv 1$. 

Another useful definition is the number of e-folds until the end of inflation, defined as 
\begin{equation}
N = -\int_t^{t_{end}} H \d t \text{\ \ \ or\ equivalently\ \ \ } \d N = -H \d t. 
\label{eqn:defN}
\end{equation}
If we let $\phi'$ denote the derivative of $\phi$ with respect to $N$, so that $\dot \phi = -H \phi'$ the Friedmann equation becomes
\begin{equation}
H^2 = \frac{V(\phi)}{3 - \tfrac{1}{2}(\phi')^2},
\end{equation}
and the Klein-Gordon equation becomes 
\begin{equation}
2H^2\phi'' + H^2(\phi')^3 - 6H^2\phi' + 2V'(\phi) = 0.
\label{eqn:KG_N}
\end{equation}

\subsection{Slow Roll Approximation}
In the slow roll approximation, we assume that $\dot \phi$ and $\ddot \phi$ are small. In order for this to be true, the inflationary potential must not be varying rapidly, leading us to the slow roll approximations:
\begin{equation}
\left | \frac{V'(\phi)}{V(\phi)} \right | <<1 \text{\ \ \ and\ \ \ } \left | \frac{V''(\phi)}{V(\phi)} \right | <<1 .
\label{eqn:SlowRollApprox}
\end{equation}
With these approximations, the Friedmann equation becomes
\begin{equation}
3 H^2 \approx_{sr} V(\phi),
\label{eqn:Friedmann_sr}
\end{equation}
and the Klein-Gordon equation becomes
\begin{equation}
3 H \dot \phi \approx_{sr} -V'(\phi)
\label{eqn:KG_sr}
\end{equation}
We use these approximations to find the initial conditions for integrating equation (\ref{eqn:KG_N}): $\phi(N=0)$ and $\dot \phi(N=0)$. To do this, we use the fact that, at the end of inflation (i.e. at $N=0$), $\epsilon = 1$. 



\bibliographystyle{apj}
\bibliography{thesis}{}

\end{document}